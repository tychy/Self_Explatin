\documentclass{jsarticle}
\usepackage{amssymb,amsmath}
\usepackage{newtxtt}
\usepackage[utf8]{inputenc}
\newcommand{\pder}[2][]{\frac{\partial#1}{\partial#2}}
\newcommand{\dder}[2][]{\frac{\mathrm{d}#1}{\mathrm{d}#2}}
\newcommand{\half}{\frac{1}{2}}
\date{\today}
\author{山田龍}
\title{n次元級の体積とガンマ関数についての計算}
\begin{document}
\maketitle
\section{n次元球体の体積}
n次元空間を考える。
今、$V_n,S_n$をそれぞれ体積、表面積とする。
一般に、$V_n = c_n r^n$と書いて、
\begin{equation}
    \dder[V_n]{r} = S_n = n c_n r^{n-1}
\end{equation}
であるとこを認める。
ここで、
\begin{align}
    I_n &= \int \exp(-(x_1^2 + x_2^2 + x_3^2 + \cdots))dx_1 dx_2 \cdots dx_n\\
        &= \pi^{\frac{n}{2}}
\end{align}
$r$を使って計算しなおせば、
\begin{align}
    I_n &= \int S_n \exp(-r^2)dr\\
        &= n c_n \int r^{n-1} \exp(-r^2)dr\\
        &= \half n c_n \int t^{\frac{n}{2}-1} \exp(-t)dt\\
        &= \half n c_n \Gamma(\frac{n}{2})\\
        &=  c_n \Gamma(\frac{n}{2} + 1)
\end{align}
したがって、
\begin{equation}
    c_n = \frac{\pi^{\frac{n}{2}}}
{\Gamma(\frac{n}{2} + 1)}\end{equation}

\section{$\Gamma(\half)$と$\Gamma(\frac{3}{2})$}
ガンマ関数の定義は、
\begin{equation}
    \Gamma(n) = \int x^{n-1} \exp(-x)dx
\end{equation}
\begin{align}
    \Gamma(\half) &= \int x^{-\half} \exp(-x)dx\\
                  &= \int u^{-1} \exp(-u^2) 2u du\\
                  &= \sqrt{\pi}
\end{align}
\begin{align}
    \Gamma(\frac{3}{2}) = \half \Gamma(\half) = \half \sqrt{\pi}
\end{align}

\end{document}

