\documentclass{jsarticle}
\usepackage{amssymb,amsmath}
\usepackage{newtxtt}
\usepackage[utf8]{inputenc}
\usepackage{braket}
\newcommand{\pder}[2][]{\frac{\partial#1}{\partial#2}}
\newcommand{\dder}[2][]{\frac{\mathrm{d}#1}{\mathrm{d}#2}}
\newcommand{\ppder}[2][]{\frac{\partial^2#1}{{\partial#2}^2}}
\newcommand{\half}{\frac{1}{2}}
\newcommand{\beq}{\begin{equation}}
\newcommand{\beql}[1]{\begin{equation}\label{#1}}
\newcommand{\eeq}{\end{equation}}
\newcommand{\eeqp}{\;\;\;.\end{equation}}
\newcommand{\eeqc}{\;\;\;,\end{equation}}
\date{\today}
\author{山田龍}
\title{平行移動演算子}
\begin{document}
\maketitle
\section{平行移動}
微小平行移動を考える。演算子$J(dx)$が変換を行うとする。
\beq
J(dx)\ket{x} = \ket{x + dx}
\eeq
任意の状態ケットに対して、
\beq
    J(dx) \ket{\alpha} = \int \ket{x+dx} \braket{x|\alpha} = \int \ket{x} \braket{x-dx|\alpha}
\eeq
この演算子に、規格化条件、和則、逆変換の存在、dxの0極限で恒等変換を要求する。
これは、
\beq
    J(dx) = 1 - i K \cdot dx
\eeq
と於けば満たされる。以下でその確認をする。Kをエルミート演算子であるとする。
また、$K$の次元は長さの逆数であるから波数の次元。
\begin{align}
    J^\dagger J &= (1 + i K^\dagger \cdot dx)(1 - i K \cdot dx)\\
                &= 1 + i (K^\dagger - K) \cdot dx + O(dx^2)\\
                &\sim 1
\end{align}
dxの二次の精度で規格化されていることがわかった。明らかに逆変換が存在する。dxの0極限での振る舞いも明らか。
\beq
J(dx + dx^\prime) = 1 - i K \cdot (dx + dx^\prime) = J(dx) J(dx^\prime)
\eeq
最後の等式もdxの二次の精度で成立。
\begin{align}
    \hat{x} J(dx) \ket{x} &= \hat{x}\ket{x+dx} = x+dx\ket{x+dx}\\
    J(dx) \hat{x} \ket{x} &= x\ket{x+dx}
\end{align}
引いて、
\beq
    [\hat{x}, J(dx)] \ket{x} = dx\ket{x}
\eeq
$J$に具体的な式を入れて計算すれば、
\beq
[\hat{x}, \hat{K}] = i
\eeq
ドブロイ仮説を持ち込んで、
\begin{align}
    \hat{p} = \frac{\hat{K}}{\hbar}\\
    [\hat{x}, \hat{p}] = i\hbar
\end{align}
\section{参考文献}
JJサクライ量子力学
\end{document}

