\documentclass{jsarticle}
\usepackage{amssymb,amsmath}
\usepackage{newtxtt}
\usepackage[utf8]{inputenc}
\newcommand{\pder}[2][]{\frac{\partial#1}{\partial#2}}
\newcommand{\dder}[2][]{\frac{\mathrm{d}#1}{\mathrm{d}#2}}
\newcommand{\half}{\frac{1}{2}}
\date{\today}
\author{山田龍}
\title{エントロピーの定義}
\begin{document}
\maketitle
\section{最大仕事とヘルムホルツの自由エネルギー}
等温準静過程において、その系が外界にする仕事の最大値を$W_{max}$と書く。
ここで熱力学第二法則:
\begin{equation}
    W_{cyc} \leqq 0
\end{equation}
$(T, X_0) \rightarrow (T, X_1) \rightarrow (T, X_0)$において、前半の過程での仕事を$-W$、
後半での仕事を等温準静過程として$W_{iq}$と書く。
熱力学第二法則より、
\begin{align}
    W - W_{iq} \leqq 0\\
    W \leqq W_{iq}
\end{align}
よって、$W_{iq}$を、最大仕事と呼び$W_{max}$と書くことにする。
いま、ある基準点における示量変数を$X_0$と書き、ヘルムホルツの自由エネルギーを
\begin{equation}
    F[T; X_1] = W_{max}((T, X_1) \rightarrow (T, X_0))
\end{equation}
と定義する。
\subsection{ヘルムホルツの自由エネルギーの性質}
\begin{equation}
    W_{max}(T,V \rightarrow V + \triangle V) = F \triangle l = p \triangle V
\end{equation}
ここで圧力は、
\begin{align}
    p &= \frac{W_{max}(T,V \rightarrow V + \triangle V)}{\triangle V} \\
      &= \frac{F[T, V] - F[T, V + \triangle V]}{\triangle V}\\ 
      &= - \pder[F]{V}
\end{align}
を得る。逆に、pをdvについて積分すればFを得る。
\section{熱}
等温過程において、$W_{max}$は断熱操作のときと違い内部エネルギーの変化のみでは記述されない。
\begin{equation}
    W_{max} = U((T, X_0) - (T, X_1)) + Q
\end{equation}
内部エネルギーの変化に対する力学的なエネルギー以外の外界とのやりとりの量として熱を上のように定義する。
\section{エントロピー}
Carnotの定理より、断熱準静的操作で繋がった状態について$(T, X_0) \rightarrow (T^\prime, X_0^\prime),$
\begin{align}
    \frac{Q_{max}(T^\prime,X_0^\prime \rightarrow X_1^\prime)}{T^\prime} =
    \frac{Q_{max}(T,X_0 \rightarrow X_1)}{T}\\
    \frac{F_(T^\prime,X_0^\prime) - F_(T^\prime,X_0^\prime) - U_(T^\prime,X_1^\prime) - U_(T^\prime,X_1^\prime)}{T^\prime}  =
    \frac{F_(T,X_0) - F_(T,X_0) - U_(T,X_1) - U_(T,X_1)}{T}\\
\end{align}
ここで
\begin{equation}
    S(T, X) = U(T, X) - F(T, X) 
\end{equation}
と定義して、
\begin{equation}
    S(T^\prime, X^\prime_1) - S(T^\prime, X^\prime_0) = S(T, X_1) - S(T, X_0)
\end{equation}
エントロピーの差が断熱準静過程で不変であることがわかった。普通はエントロピーは断熱静的過程で不変であるように定義される。
\end{document}

