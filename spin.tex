\documentclass{jsarticle}
\usepackage{amssymb,amsmath}
\usepackage{newtxtt}
\usepackage[utf8]{inputenc}
\usepackage{braket}
\usepackage{bm}
\newcommand{\pder}[2][]{\frac{\partial#1}{\partial#2}}
\newcommand{\dder}[2][]{\frac{\mathrm{d}#1}{\mathrm{d}#2}}
\newcommand{\half}{\frac{1}{2}}
\newcommand{\beq}{\begin{equation}}
\newcommand{\beql}[1]{\begin{equation}\label{#1}}
\newcommand{\eeq}{\end{equation}}
\newcommand{\eeqp}{\;\;\;.\end{equation}}
\newcommand{\eeqc}{\;\;\;,\end{equation}}
%量子力学用
\newcommand{\hr}{\hat{r}}
\newcommand{\hp}{\hat{p}}
\newcommand{\hj}{\hat{j}}
\newcommand{\hjm}{\hat{j}^{-}}
\newcommand{\hjp}{\hat{j}^{+}}
\newcommand{\hbj}{\hat{\bm{j}}}
\newcommand{\hbjt}{\hat{\bm{j^2}}}
\newcommand{\hcj}{\hat{J}}
\newcommand{\hcp}{\hat{P}}
\newcommand{\rh}{\frac{1}{i\hbar}}
\date{\today}
\author{山田龍}
\title{Spin}
\begin{document}
\maketitle
\section{生成子}
回転並進変換を考える。
\beq
    r_i \rightarrow M r_i + a
\eeq
演算子に対して
\beq
\hat{r}_i \rightarrow M \hat{r_i} + a
\eeq
とするようなユニタリー変換を考えたい。
\begin{align}
    U^\dagger \hr U &= M r + a\\
    U^\dagger \hp U &= Mp\\
\end{align}
となる$U$を導入すれば、ユニタリー変換された状態ベクトル$U\Ket{\psi}$について期待値は古典の場合と同じように変換される。
ここで、具体的にユニタリー変換を構成する。
無限小回転を
\beq
M(\theta) = \bm{x} + \bm{x}  \times \theta
\eeq
ユニタリー演算子を以下のように書けば、
\beq
U(M, \epsilon) = 1 + \rh \theta \cdot \hat{J} + \rh \epsilon \cdot \hat{P}
\eeq
$\hcj ,\hcp$が満たすべき関係式は
\begin{align}
    \rh [\hr ,\theta \cdot \hat{J}] = \theta \times \bm{r}\\
    \rh [\hp ,\theta \cdot \hat{J}] = \theta \times \bm{p}\\
\rh [\hr ,\epsilon \cdot \hcp] = \epsilon\\
\rh [\hp ,\epsilon \cdot \hcp] = 0\\
\end{align}
実際には系を記述する変数は更にスピン自由度がある。ここでは系が$\hr ,\hp$で全てかかれるとして、
\begin{align}
    \hcp &= \sum \hp\\
    \hcj &= \sum \hr \times \hp
\end{align}
最後に$\hcj ,\hcp$の交換関係について、
\begin{align}
    [J^a,J^b] = ih \sum \epsilon_{abc}J^c\\
[J^a,P^b] = ih \sum \epsilon_{abc}P^c\\
[P^a,P^b] = 0
\end{align}
\section{角運動量の固有状態}
以下ではbold体の肩についている数字はべきを表す。それ以外は要素。
角運動量演算子を因子化して、
\beq
    \hcj = \hbar \hbj
\eeq
\beq
[\hj^a,\hj^b] = \sum \epsilon_{abc}\hj^c\\
\eeq
$\hbj$はエルミートである。$\hbj$の成分は交換しないので、全てを同時に対角化することはできない。
$\hj^3$について対角化する。
まず、
\begin{align}
    \hj^{\pm} = \hj^1 \pm i \hj^2
\end{align}
のように、$\hj^{1,2}$を書き換える。これらははしご演算子と呼ばれる非エルミート演算子である。
すると、
\beq
[\hbjt,\hj^3] = 0
\eeq
から、同時対角化できることがわかる。
$\hj^{\pm},\hj^3$の交換関係を計算する。
\begin{align}
    [\hj^3, \hj^{+}] &= \hj^{+}\\
[\hj^3, \hj^{-}] &= -\hj^{-}\\
[\hj^{+}, \hj^{-}] &= 3\hj^{3}
\end{align}
また、
\beq
\hbjt = \half (\hjp\hjm + \hjm\hjp) + (\hj^3)^2 =  \hjm\hjp + (\hj^3)^2 + \hj^3
\eeq
がわかる。
ここから、同時対角化する基底を使って$\hj^3$の固有値を$j$とおく。
\beq
    (\hbj)^2 \ket{\psi} = j(j+1)\ket{\psi}
\eeq
\section{スピン}
\section{角運動量合成}
\section{電子のスピン合成}
\end{document}

