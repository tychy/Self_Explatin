\documentclass{jsarticle}
\usepackage{amssymb,amsmath}
\usepackage{newtxtt}
\usepackage[utf8]{inputenc}
\usepackage{braket}
\newcommand{\pder}[2][]{\frac{\partial#1}{\partial#2}}
\newcommand{\dder}[2][]{\frac{\mathrm{d}#1}{\mathrm{d}#2}}
\newcommand{\half}{\frac{1}{2}}
\newcommand{\beq}{\begin{equation}}
\newcommand{\beql}[1]{\begin{equation}\label{#1}}
\newcommand{\eeq}{\end{equation}}
\newcommand{\eeqp}{\;\;\;.\end{equation}}
\newcommand{\eeqc}{\;\;\;,\end{equation}}
%量子力学用
\newcommand{\hr}{\hat{r}}
\newcommand{\hp}{\hat{p}}
\date{\today}
\author{山田龍}
\title{Spin}
\begin{document}
\maketitle
\section{生成子とパリティ変換}
回転並進変換を考える。
\beq
    r_i \rightarrow M r_i + a
\eeq
演算子に対して
\beq
\hat{r}_i \rightarrow M \hat{r_i} + a
\eeq
とするようなユニタリー変換を考えたい。
\begin{align}
    U^\dagger \hr U &= M r + a\\
    U^\dagger \hp U &= Mp\\
\end{align}
となる$U$を導入すれば、ユニタリー変換された状態ベクトル$U\Ket{\psi}$について期待値は古典の場合と同じように変換される。

\section{スピン}
ask
\end{document}

